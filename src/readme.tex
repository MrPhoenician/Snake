\documentclass[10pt, oneside]{beamer}

\begin{document}
    \title{BrickGame ver 2.0}
    \author{by harkonex}
    \date{13.01.2025}
    \begin{frame}
        \titlepage
    \end{frame}

%    \setbeamercolor{itemize item}{fg=black} % черные треугольники
    \setbeamertemplate{itemize items}{\textbullet}% отключает синие треугольники

    \begin{frame}
        BrickGame is a popular handheld console from the 90s with several thousands of built-in games developed in China.
        It was originally a copy of Tetris developed in the USSR and released by Nintendo as part of the GameBoy
        platform, but also included many other games that were added over time. The console had a small screen with a
        10x20 playing field that was a matrix of "pixels". To the right of the field was a scoreboard with a digital
        display of the current game status, records, and other additional information. The most popular games on
        BrickGame were Tetris, Tanks, Racing, Frogger and Snake.
    \end{frame}

    \begin{frame}
        Snake.
        The player controls a snake that moves forward continuously. The player changes the direction of the snake by
        using the arrows. The goal of the game is to collect "apples" that appear on the playing field. The player must
        avoid hitting the walls of the playing field. After "eating" the next "apple", the length of the snake increases
        by one. The player wins when the snake reaches the maximum size (200 "pixels"). If the snake hits a boundary of
        the playing field, the player loses.
        The game was based on another game called Blockage. It had two players controlling characters that left a trail
        that you couldn't run into. The player who lasted the longest won. In 1977, Atari released Worm, which was now a
        single player game. The most popular version of the game is probably the 1997 version released by the Swedish
        company Nokia for their Nokia 6110 phone, developed by Taneli Armanto.
    \end{frame}

    \begin{frame}
        \color{blue} To install use the following commands:
        \begin{itemize}
            \item make install - program installation
            \item make uninstall - program uninstallation
        \end{itemize}
    \end{frame}

    \begin{frame}
        \color{blue} To control use the following keys:
        \begin{itemize}
            \item S - Start game;
            \item Q - End game;
            \item P - Pause;
            \item Left arrow - move to the left;
            \item Right arrow - move to the right;
            \item Down arrow - move to the down;
            \item Up arrow - move to the up;
            \item Space - acceleration;
        \end{itemize}
    \end{frame}

    \begin{frame}
        \color{blue} Points are awarded as follows:
        \begin{itemize}
            \item 1 apple = +1 point;
            \item 5 points = +1 lvl;
            \item 1 lvl = +1 speed;
        \end{itemize}
    \end{frame}

\end{document}